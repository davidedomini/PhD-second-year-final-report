% This is samplepaper.tex, a sample chapter demonstrating the
% LLNCS macro package for Springer Computer Science proceedings;
% Version 2.21 of 2022/01/12
%
\documentclass[runningheads]{llncs}
\usepackage[inline]{enumitem}
\usepackage[utf8]{inputenc}
\usepackage{amsmath}
\usepackage{geometry}
\geometry{a4paper, left=25mm, right=25mm, top=25mm, bottom=25mm}

% \usepackage{fontspec}
%
\usepackage[T1]{fontenc}
% T1 fonts will be used to generate the final print and online PDFs,
% so please use T1 fonts in your manuscript whenever possible.
% Other font encondings may result in incorrect characters.
%
\usepackage{graphicx}
% Used for displaying a sample figure. If possible, figure files should
% be included in EPS format.
%
% If you use the hyperref package, please uncomment the following two lines
% to display URLs in blue roman font according to Springer's eBook style:
%\usepackage{color}
%\renewcommand\UrlFont{\color{blue}\rmfamily}
%


\begin{document}

\begin{titlepage}
    \centering
    \vspace*{2cm}
    
    {\scshape\Large PhD Programme in Computer Science and Engineering \par}
    \vspace{0.5cm}
    {\scshape\large Cycle XXXIX \par}
    \vspace{0.5cm}

    \rule{\linewidth}{0.4mm} \\ [0.1mm]
    \raisebox{0.2cm}{\rule{\linewidth}{0.8mm}} \\[0.8cm]
    {\huge\bfseries PhD First Year -- Final Report \par}
    \vspace{0.8cm}
    \rule{\linewidth}{0.8mm} \\ [0.1pt]
    \raisebox{0.2cm}{\rule{\linewidth}{0.4mm}} \\[1.5cm]
    
    % {\Large PhD First Year -- Final Report \par}

    \vspace{1.5cm}
    
    \noindent
    \begin{minipage}[t]{0.45\textwidth}
        \raggedright
        \textbf{Commission:}\\[0.5cm]
        Prof. Mirko Viroli\\
        Prof. Danilo Pianini\\
        Prof. Matteo Ferrara
    \end{minipage}%
    \hfill
    \begin{minipage}[t]{0.45\textwidth}
        \raggedleft
        \textbf{PhD Student:}\\[0.5cm]
        Davide Domini
    \end{minipage}
    
    \vfill
    
\end{titlepage}

\section{Context}
Computing devices have become ubiquitous in everyday life.
%
This trend has paved the way for the emergence of a field known as collective computing~\cite{DBLP:journals/computer/Abowd16}.
%
These systems are composed of many -- potentially heterogeneous -- interacting devices. 
%
Each device is equipped with sensors to perceive the surrounding environment and is able to communicate with a subset of other devices 
 (e.g., based on logical or physical proximity). %, thus creating a neighborhood.
%
Some real-world examples of these systems include: 
 crowd management using wearable devices or smartphones,
 smart surveillance using flocks of drones, 
 air quality monitoring, and many more. 
%
Analyzing collective systems it is possible to identify some key features, namely:
\begin{enumerate*}[label=(\roman*)]
    \item each device, potentially, produces a huge amount of data;
    \item devices need to cooperate among themselves to solve the collective task; and
    %\item data are naturally distributed (i.e., ); and
    \item data are subject to privacy constraints due to laws (e.g., GDPR~\cite{GDPR} in the European Union).
\end{enumerate*}

In recent years, macroprogramming approaches~\cite{DBLP:journals/csur/Casadei23} have been proposed to shift the focus onto the collective of devices
 rather than on the single device.
%
More in particular, Aggregate Computing (AC)~\cite{DBLP:journals/computer/BealPV15} proposes the development of a single aggregate program as a 
 composition of functions over computational fields (i.e., an abstraction that maps each device to a value in space and time). 
%
AC is rooted in Field Calculus (FC)~\cite{DBLP:conf/forte/AudritoVDPB19}, a core language designed to capture the essential aspects of languages that make use of
 computational fields and that allows the formal demonstration of properties such as 
 self-stabilization~\cite{DBLP:conf/coordination/ViroliD14} and eventual consistency~\cite{DBLP:journals/taas/BealVPD17}.
%
Moreover, AC allows the definition of scalable and resilient programs, thus being a good candidate as a framework 
 for cooperative learning in collective systems.
%
%While AC allows the definition of scalable and resilient programs, encoding very complex behaviors is still an open challange.
%
For this reason, attempts have been made to combine AC and machine learning, first using multi-agent reinforcement learning 
techniques~\cite{DBLP:conf/coordination/AguzziCV22,DBLP:conf/acsos/AguzziVE23,DBLP:conf/coordination/DominiCAV23,DBLP:journals/scp/DominiCAV24,DBLP:conf/woa/DominiFAV24}, 
and then shifting towards federated learning~\cite{DBLP:conf/coordination/DominiAEV24,DBLP:conf/acsos/DominiFAVE24}. 

%
Specifically, Federated Learning (FL)~\cite{DBLP:conf/aistats/McMahanMRHA17} has emerged as a framework to cooperatively train a joint model
 while maintaining data localization.
%
Classic FL approaches are based on a simple client-server architecture in which each client $c_i$, at each time step $t$, is responsible for training 
 a model $m_i^t$ on the local dataset $D_i$ (i.e., the local model), while the server is responsible for collecting all the 
 models $\{m_1^t, \dots, m_n^t\}$ and for aggregating them -- following some aggregation policy~\cite{DBLP:journals/fgcs/QiCGIFP24} -- into the 
 new global model $m_g^{t+1}$ which will be further locally improved by each device at the next time step.
%
Since FL does not require sharing the data it is possible to work also in contexts with privacy concerns and where it is not possible to collect
 all the data in a centralized server due to their huge volume and resource constraints.
%
However, two main challenges are still open, namely:
\begin{enumerate*}[label=(\roman*)]
    \item in real-world systems, data are often non-independently and identically distributed (non-iid), thus leading centralized implementations
     of FL to convergence issues~\cite{DBLP:conf/mir/YangLHSL024}; and
    \item relying on a single central server introduces a bottleneck (especially in systems with a huge amount of devices) 
     and a single point of failure (i.e., if the server fails then all the learning process is interrupted).
\end{enumerate*}

To tackle these challenges, in the literature, works have been proposed to introduce FL algorithms based 
 on distributed architectures~\cite{DBLP:journals/jpdc/HegedusDJ21,DBLP:conf/dsn/WinkN21}
 and personalization~\cite{DBLP:journals/tnn/TanYCY23}.
%
More in particular, our focus lies on a subfield of personalized FL which assumes that devices can be clustered into groups and that inside each 
 group data are independently and identically distributed (iid)~\cite{DBLP:journals/tit/GhoshCYR22,DBLP:conf/ecai/Li0TWXZ23}, 
 thus avoiding convergence issues and producing models with better performance.
% 
Although clustered FL generally has significant potential, its applicability in the context of collective systems holds many challenges, mainly: 
 clustering is performed in a centralized manner, and a fixed number of clusters must be chosen a priori. 
% 
This causes issues because, in contexts with numerous devices, there is considerable overhead associated with selecting clusters. 
%
Moreover, predefining a number of clusters is often a difficult task, and an incorrect number can lead to convergence problems.
%
Finally, clustered FL methods in the literature does not take into account the spatial distribution of devices.
%
However, in many contexts, this information can be crucial for achieving better clustering. 
%
For instance, consider a task that uses text written by users on their phones; it is very likely that users who are 
 geographically close (e.g., within the same state) will write similarly, whereas users who are far apart 
 will produce data with different distributions.

\section{Methodology and Preliminary Results}\label{sec:methodology}

 Initially, a review of related works in the literature has been conducted to assess the current state of development in FL 
  and to identify key challenges.
 %
 Specifically, the focus was on subareas relevant to our interests, namely: 
  higly distributed systems and non-iid data.
  %\dom{forse higly distributed systems va cambiato con qualcosa di più specifico}
 %
 Once identified the main open problems and challenges in FL we started to design custom algorithms based 
  on field coordination.
  Regarding AC, we have exploited the ScaFi~\cite{DBLP:journals/softx/CasadeiVAP22} implementation and toolchain
  which provides a high-level functional DSL in Scala.
  On the other hand, as deploying such systems in real-world has high costs, we used a simulator for pervasive systems, 
  namely: Alchemist~\cite{DBLP:journals/jos/PianiniMV13}.
 
 More in detail -- in our first work~\cite{DBLP:conf/coordination/DominiAEV24} -- we tried to assess two different aspects of
  FL based on field coordination.
 %
 On the one hand, we implemented a full peer-to-peer (i.e., devices are not split into clusters) FL algorithm using AC abstractions, then we compared it with a 
  centralized implementation of FL to see whether results and accuracies were comparable.
 %
 On the other hand, we implemented a version of FL based on the SCR pattern. 
 %
 The hypothesis behind this experiment was that geographically close devices may have a similar data 
  distribution for a certain phenomenon.
 % 
 Exploiting SCR allowed us to create multiple sub-zones within a certain range and then perform a centralized FL inside each zone,
  thus evolving multiple specialized models.
 %
 To experiment with this algorithm we used PM10 samples from different stations in Europe\footnote{Dataset available at: \url{https://www.eea.europa.eu/data-and-maps/dashboards/air-quality-statistics}}
  and performed a task of air quality forecasting.
 %
 Results have shown that the proposed approach has similar performance to the centralized implementation when using iid data 
  while outperforming it in the case of non-iid data with distributions influenced by the geographical location of devices.
 
 Finally, we proposed another approach~\cite{DBLP:conf/acsos/DominiFAVE24} which extends the implementation based on the SCR pattern
     using the concept of space-fluidity~\cite{DBLP:journals/lmcs/CasadeiMPVZ23}.
 %
 In this way, it is possible to create specialized subzones that are not only based on spatial proximity but also take into account other 
  specified metrics.
 %
 For instance, we used a metric based on the loss of local models to broaden or shrink zones to match a specified 
  maximum loss threshold.   

\section{Future Work and Research Plan}\label{sec:future}

Beyond the preliminary results obtained, we believe that the integration of federated learning and macroprogramming for collective systems still has several
 open directions that can lead to improvements. 

Firstly, considering the context of collective systems and the limited resources available to various devices in most cases, 
%we believe that an interesting direction could be
 a particularly promising direction that will be pursued in the near future is 
 experimenting with methods like quantization and pruning or using novel architectures like the sparse neural networks, 
 which aim to limit the use of weights in order to reduce the computational load required for training and inference.

Moreover, it would be interesting to study the effect that the movement of nodes in space and consequently between various clusters has. 
%
We believe that this, within certain limits, can bring benefits due to the sharing of previously acquired knowledge (i.e., knowledge fertilization). 
%
To study this phenomenon, metrics will need to be defined to quantitatively and objectively measure the effects.

Finally, given the huge variability over time of the environments in which these systems are immersed, a solution to maintain adaptability over time could 
 be to integrate the proposed solutions with continual learning.

\section{Pubblications}
List of published (or already accepted) papers:
\begin{enumerate}
    \item \emph{ScaRLib: A Framework for Cooperative Many Agent Deep Reinforcement Learning in Scala}~\cite{DBLP:conf/coordination/DominiCAV23} \\
    \textbf{Abstract: }
    Multi Agent Reinforcement Learning (MARL) is an emerging field in machine learning where multiple agents learn, simultaneously and in a shared environment, 
     how to optimise a global or local reward signal. MARL has gained significant interest in recent years due to its successful applications in various domains,
     such as robotics, IoT, and traffic control. Cooperative Many Agent Reinforcement Learning (CMARL) is a relevant subclass of MARL, where thousands of 
     agents work together to achieve a common coordination goal.
    %
    In this paper, we introduce ScaRLib, a Scala framework relying on state-of-the-art deep learning libraries to support the development of CMARL systems. 
    %
    The framework supports the specification of centralised training and decentralised execution, and it is designed to be easily extensible, allowing to add new 
     algorithms, new types of environments, and new coordination toolchains.
    %
    This paper describes the main structure and features of ScaRLib and includes basic demonstrations that showcase binding with one such toolchain: 
     ScaFi programming framework and Alchemist simulator can be exploited to enable learning of field-based coordination policies for large-scale systems.
    %
    \item \emph{Field-Based Coordination for Federated Learning}~\cite{DBLP:conf/coordination/DominiAEV24} \\
    \textbf{Abstract: }
    Federated Learning has gained increasing interest in the last years, as it allows the training of machine learning models with a large number of devices 
     by exchanging only the weights of the trained neural networks. 
    % 
    Without the need to upload the training data to a central server, privacy concerns and potential bottlenecks can be removed as fewer data is transmitted. 
    %
    However, the current state-of-the-art solutions are typically centralized, and do not provide for suitable coordination mechanisms to take into account 
     spatial distribution of devices and local communications, which can sometimes play a crucial role. 
    % 
    Therefore, we propose a field-based coordination approach for federated learning, where the devices coordinate with each other through the use of 
     computational fields. 
    %
    We show that this approach can be used to train models in a completely peer-to-peer fashion. 
    %
    Additionally, our approach also allows for emergently create zones of interests, and produce specialized models for each zone enabling each agent 
     to refine their models for the tasks at hand.
    %
    We evaluate our approach in a simulated environment leveraging aggregate computing—the reference global-to-local field-based coordination programming paradigm. 
    %
    The results show that our approach is comparable to the state-of-the-art centralized solutions, while enabling a more flexible and scalable approach 
     to federated learning.
    \item \emph{ScaRLib: Towards a hybrid toolchain for aggregate computing and many-agent reinforcement learning}~\cite{DBLP:journals/scp/DominiCAV24} \\
    \textbf{Abstract: }
    This article introduces ScaRLib, a Scala-based framework that aims to streamline the development cyber-physical swarms scenarios 
     (i.e., systems of many interacting distributed devices that collectively accomplish system-wide tasks) by integrating macroprogramming and multi-agent 
     reinforcement learning to design collective behavior. 
    %
     This framework serves as the starting point for a broader toolchain that will integrate these two approaches at multiple points to harness the 
     capabilities of both, enabling the expression of complex and adaptive collective behavior.

    \item \emph{Towards Intelligent Pulverized Systems: a Modern Approach for Edge-Cloud Services}~\cite{DBLP:conf/woa/DominiFAV24} \\
    \textbf{Abstract: }
    Emerging trends are leveraging the potential of the edge-cloud continuum to foster the creation of smart services capable of adapting to the 
     dynamic nature of modern computing landscapes. 
    %
     This adaptation is achievable through two primary methods: by leveraging the underlying architecture to refine machine learning algorithms, 
      and by implementing machine learning algorithms to optimize the distribution of resources and services intelligently. 
    %  
    This paper explores the latter approach, focusing on recent advancements in pulverized architecture, collective intelligence, and many-agent 
     reinforcement learning systems. 
    %
    This novel trend, which we refer to as intelligent pulverized system (IPS), aims to create a new generation of services that can adapt to the 
     complex and dynamic nature of the edge-cloud continuum. Our proposed learning framework integrates many-agent reinforcement learning, graph neural 
     networks, and aggregate computing to create intelligent services tailored for this environment. 
    %
    We discuss the application of this framework across different levels of the pulverization model, illustrating its potential to enhance 
     the adaptability and efficiency of services within the edge-cloud continuum.
    \item \emph{Proximity-based Self-Federated Learning}~\cite{DBLP:conf/acsos/DominiFAVE24} \\
    \textbf{Abstract: }
    In recent advancements in machine learning, federated learning allows a network of distributed clients to collaboratively develop a global model 
     without needing to share their local data. 
    %
    This technique aims to safeguard privacy, countering the vulnerabilities of conventional centralized learning methods. 
    %
    Traditional federated learning approaches often rely on a central server to coordinate model training across clients, aiming to replicate 
     the same model uniformly across all nodes. 
    %
    However, these methods overlook the significance of geographical and local data variances in vast networks, potentially affecting model 
     effectiveness and applicability. 
    % 
    Moreover, relying on a central server might become a bottleneck in large networks, such as the ones promoted by edge computing. 
    %
    Our paper introduces a novel, fully-distributed federated learning strategy called proximity-based self-federated learning that enables 
     the self-organised creation of multiple federations of clients based on their geographic proximity and data distribution without 
     exchanging raw data.
    %
    Indeed, unlike traditional algorithms, our approach encourages clients to share and adjust their models with neighbouring nodes based on geographic 
     proximity and model accuracy. 
    %
    This method not only addresses the limitations posed by diverse data distributions but also enhances the model's adaptability to different regional 
     characteristics creating specialized models for each federation. We demonstrate the efficacy of our approach through simulations on 
     well-known datasets, showcasing its effectiveness over the conventional centralized federated learning framework.

    \item \emph{Towards Self-Adaptive Cooperative Learning in Collective Systems}~\cite{DBLP:conf/acsos/Domini24} \\
    \textbf{Abstract: }
    Nowadays, collective adaptive systems have become crucial as modern systems increasingly adopt this vision.
    %
    These systems can be leveraged as a means to facilitate cooperative adaptive learning.
    %
    However, implementing such systems presents various challenges, including:
     scalability, failures, non-iid data and complex architectures.
    %
    This paper presents a modern approach to cooperative and privacy-resilient learning by leveraging macroprogramming.
    %
    Specifically, we propose a new framework based on the integration of aggregate computing and federated learning,
     aiming to address these challenges and enhance the effectiveness and security of cooperative learning systems.

    \item \emph{A Reusable Simulation Pipeline for Many-Agent Reinforcement Learning}~\cite{DBLP:conf/dsrt/DominiAPV24} \\
    \textbf{Abstract: }
    Recent advancements in multi-agent reinforcement learning led to systems in which large groups of agents
     work together to learn shared policies and achieve collective behavior.
    %
    This approach is increasingly important for many applications,
     including swarm robotics, crowd sensing, and large-scale IoT networks.
    %
    In fact, these systems require repeated experimentation to learn from experience:
     simulation becomes thus essential, as deploying and testing in real-world environments incurs in high costs and practical challenges.
    %
    In response to this need, our paper introduces a simulation-based pipeline to gather the necessary experience for many-agent learning.
    %
    We highlight the requirements of such pipeline and the role of simulation, presenting also a practical prototype implemented in Alchemist,
     a simulator designed for very large-scale systems.
    %
    This pipeline provides a scalable, modular, and flexible environment for developing and testing many-agent reinforcement learning strategies.
\end{enumerate}

\section{Attended Conferences and Workshops}
\begin{enumerate}
    \item \textbf{Coordination Models and Languages} - 26th International Conference, COORDINATION 2024, Held as Part of the 19th International Federated Conference on Distributed Computing Techniques, DisCoTec 2024, Groningen, The Netherlands, June 17-21, 2024
    \item \textbf{25th Workshop "From Objects to Agents"}, Bard (Aosta), Italy, July 8-10, 2024
    \item \textbf{IEEE International Conference on Autonomic Computing and Self-Organizing Systems}, ACSOS 2024, Aarhus, Denmark, September 16-20 2024
    \item \textbf{11th Workshop on Self-Improving Systems Integration}, SISSY 2024, Held as Part of ACSOS 2024, Aarhus, Denmark, September 16-20 2024
    \item \textbf{2nd International Workshop on Artificial Intelligence for Autonomous computing Systems}, AI4AS 2024, Held as Part of ACSOS 2024, Aarhus, Denmark, September 16-20 2024
    \item \textbf{28th International Symposium on Distributed Simulation and Real Time Applications}, DS-RT 2024, Urbino, Italy, October 7-9 2024
\end{enumerate}

\section{Doctoral Schools}
\begin{enumerate}
    \item \textbf{Bertionoro International Spring School 2024}, BISS 2024, Bertinoro, Italy, March 11-15 ,2024
\end{enumerate}

\section{Teaching Tutor}
\begin{enumerate}
    \item \textbf{Machine Learning Systems for Data Science} - Statistical Science Bachelor Degree
    \item \textbf{Software Engineering (Modulo 1)} - Digital Transformation Management Master Degree
\end{enumerate}

\section{PhD Courses}

\begin{enumerate}
    \item \textbf{BISS 2024 - Bertinoro International Spring School - Program Analysis}
    \begin{itemize}
        \item Prof: Roberta Gori, Roberto Bruni  - Università di Pisa
        \item Proposed CFU: 4 (12 hours) 
        \item Period: March 2024
        \item Exam: Not done, CFUs without evaluation
    \end{itemize}    
    \item \textbf{BISS 2024 - Bertinoro International Spring School - Graph Neural Networks}
    \begin{itemize}
        \item Prof: Fabrizio Silvestri - Sapienza Università di Roma
        \item Proposed CFU: 4 (12 hours) 
        \item Period: March 2024
        \item Exam: Done
    \end{itemize}  
    \item \textbf{BISS 2024 - Bertinoro International Spring School - Large Language Models}
    \begin{itemize}
        \item Prof: Danilo Croce - University of Roma, Tor Vergata
        \item Proposed CFU: 4 (12 hours) 
        \item Period: March 2024
        \item Exam: Done
    \end{itemize}  
    \item \textbf{Risk Assessment of ML for Cybersecurity}
    \begin{itemize}
        \item Fabio Pierazzi - King's College London
        \item Proposed CFU: 5 (20 hours) 
        \item Period: April 2024
        \item Exam: Done
    \end{itemize}    
    \item \textbf{Introduction to Complex Systems Science} 
    \begin{itemize}
        \item Prof: Andrea Roli - Università di Bologna      
        \item Proposed CFU: 2 (10 hours) 
        \item Period: June 2024
        \item Exam: To do
    \end{itemize}    
    
\end{enumerate}
%
% ---- Bibliography ----
%
% BibTeX users should specify bibliography style 'splncs04'.
% References will then be sorted and formatted in the correct style.
%
\bibliographystyle{alpha}
\bibliography{mybibliography}
%
% \begin{thebibliography}{8}
% \bibitem{ref_article1}
% Author, F.: Article title. Journal \textbf{2}(5), 99--110 (2016)

% \bibitem{ref_lncs1}
% Author, F., Author, S.: Title of a proceedings paper. In: Editor,
% F., Editor, S. (eds.) CONFERENCE 2016, LNCS, vol. 9999, pp. 1--13.
% Springer, Heidelberg (2016). \doi{10.10007/1234567890}

% \bibitem{ref_book1}
% Author, F., Author, S., Author, T.: Book title. 2nd edn. Publisher,
% Location (1999)

% \bibitem{ref_proc1}
% Author, A.-B.: Contribution title. In: 9th International Proceedings
% on Proceedings, pp. 1--2. Publisher, Location (2010)

% \bibitem{ref_url1}
% LNCS Homepage, \url{http://www.springer.com/lncs}. Last accessed 4
% Oct 2017
% \end{thebibliography}
\end{document}
